\documentclass[runningheads]{llncs}
%
\usepackage{graphicx}
\usepackage[ngerman]{babel}
\usepackage{cite}
\usepackage{array}
% \usepackage{hyperref}
% \renewcommand\UrlFont{\color{blue}\rmfamily} % für hyperref

\begin{document}

\title{Digitale Transformation der Arbeitswelt: Auswirkungen von Remotearbeit auf Ökonomie, Ökologie und Soziales}

\titlerunning{Digitale Transformation der Arbeitswelt}

\author{Fabian Klimpel\inst{1}\orcidID{} \and
Tobias Tögel\inst{1}\orcidID{} \and
Johannes Brandenburger\inst{1}\orcidID{}}
%
\authorrunning{Klimpel, Tögel, Brandenburger}

\institute{HTWG - Hochschule Konstanz Technik, Wirtschaft und Gestaltung}

\maketitle

\begin{abstract}
The abstract should briefly summarize the contents of the paper in
15--250 words.

\keywords{First keyword  \and Second keyword \and Another keyword.}

\end{abstract}

\section{Verbreitung von Remotearbeit}
Laut Ifo Institut arbeiteten im Jahr 2023 Gesamtwirtschaftlich betrachtet 24,7\% aller Beschäftigten in Deutschland teileweise im Homeoffice\cite{ifo23}. Einen besonders hohen Anteil hat der Dienstleistungssektor mit 35,6\%\cite{ifo23}.Dienstleistungen der Informationstechnologie haben mit 73.4\% wiederrum den höchsten Anteil im Dienstleistungssektor\cite{ifo23}.\\
Entwicklungen der letzten 5 Jahre zeigen, dass durch die Covid-19 Pandemie das Arbeiten von Zuhause stark an Bedeutung gewonnen hat. Zwischen den Jahren 2019 \& 2020 hat sich der Wert von 13\% auf 21\% fast verdoppelt\cite{statis23}. Laut Ifo Institut gab es nach Aufhebung der Homeofficepflicht im März 2022 keine Veränderungen des Anteils an Beschäftigten, die teileweise im Homeoffice arbeiten\cite{ifo23}.

\section{Intergenerationale Aspekte}

\subsection{Historische Entwicklung von Remotearbeit}
Erste Telearbeitkonzepte entstanden in den 1970er Jahren in den USA\cite{Dangelmaier.1999}. Anfang 1980er taucht der Begriff Telearbeit erstmals in Deutschland auf mit Pilot\-projekten bei Siemens\cite{Dangelmaier.1999}.Aufgrund der hohen ökonomischen \& technologischen Barrieren gewann Telearbeit erst im laufe der Zeit an Relevanz\cite{Dangelmaier.1999}.\\
In den 1990er Jahren entstanden neue Konzepte für Organisationsformen wie beispielsweise das Konzept "Virtuelle Organisation"\cite{Siedenbiedel.2020}. Durch das Entstehen und die rasante Entwicklung moderner IuK und der damit verbundenen Möglichkeiten durch Telearbeit, wurde die virtuelle Organisation erst möglich\cite{Siedenbiedel.2020}.

\subsection{Intergenerationaler Vergleich}
Bei einem Vergleich zwischen heutigen Berufseinsteigern und früheren Generationen lassen sich drastische Unterschiede in der Arbeitswelt erkennen:
\vspace{-\baselineskip}
\begin{center}
\begin{tabular}{ | m{13em} | m{13em}| m{13em} | } 
\hline
Generation (*1928-*1945) & Generation (*1946-*1964) & Generation ($>$*1996) \\
\hline
Pre-Remotearbeit & erste Generation mit Remotearbeitsmöglichkeiten & Remotearbeit Teil des Arbeits\-alltags\\
\hline
- & Geringe Rechenleistung in stationären Geräten & Große Rechenleistung in tragbaren Geräten\\
\hline
- & Verfügbarkeit des Internet ab 1993\cite{CERN} & Verfügbarkeit von Highspeed-Internet\\
\hline
- & Anfangs weltweit wenige tausend Telearbeiter\cite{heilmann1983softwareentwicklung} & in Deutschland ca. 11,5 mio. Beschäftigte, teilweise im Homeoffice (2023)\cite{statis24, ifo23}\\
\hline
\end{tabular}
\end{center}

% \section{Technologische Voraussetzungen und Herausforderungen}

% \subsection{Infrastruktur und Technologien für erfolgreiche Remotearbeit}

% \subsection{Datenschutz und Sicherheitsaspekte}

% \subsection{Umgang mit technologischen Herausforderungen}


\section{Ökonomische Auswirkungen}

% Die Auswirkungen von Remotearbeit auf die Wirtschaft und Unternehmen sind sehr vielseitig.
% Im Folgenden werden die zwei wichtigsten Aspekte, die Einsparungen und die Produktivitätsveränderung, genauer betrachtet.

\subsection{Einsparungen durch Remotearbeit}

% Die Einsparungen durch Remotearbeit können in verschiedenen Bereichen auftreten.
Eine umfassende Studie vom Capgemini Research Institute ergab, dass Unternehmen durch Remotearbeit allein die \textbf{Immobilienkosten} um 36 \% zu reduzieren erwarten \cite{capgemini_research_institute_future_2020}.
88 \% der befragten Unternehmen gaben an, bereits jetzt Einsparungen in diesem Bereich zu verzeichnen \cite{capgemini_research_institute_future_2020}.
Weitere Einsparungen können in den \textbf{Fahrkosten}, genauer den Kosten für das Pendeln, erzielt werden.
So sparen Hybridarbeiter am Tag durchschnittlich 9,11 Dollar, wenn Sie von zu Hause aus statt im Büro arbeiten \cite{owl_labs_state_2022}.
Auch Arbeitgeber profitieren von Einsparungen im Verkehrssektor:
Die Kosten für \textbf{Geschäftsreisen} sollen um 26 bis 45 \% sinken \cite{capgemini_research_institute_future_2020}.

\subsection{Produktivitätsveränderung durch Remotearbeit}

Ob die Produktivität durch Remotearbeit steigt oder sinkt, ist sehr umschritten.
Einige Studien zeigen einen Anstieg der Produktivität von 19 bis 32 \% \cite{capgemini_research_institute_future_2020}, \cite{glenn_dutcher_effects_2012}, \cite{owl_labs_state_2022}, \cite{global_workplace_analytics_latest_2021}, andere einen Rückgang von 6 bis 12 \% \cite{emanuel_working_2023}, \cite{steven_j_davis_evolution_2023}.
Gründe für eine positive Auswirkung auf die Produktivität sind vor allem flexiblere Arbeitszeiten, weniger Pendelzeit und weniger Ablenkungen durch Kollegen \cite{capgemini_research_institute_future_2020}.
Negative Auswirkungen auf die Produktivität haben vor allem die fehlende direkte Interaktion und Ablenkungen durch das private Umfeld \cite{holand_homeoffice_2023}, \cite{emanuel_working_2023}.

Prinzipiell lässt sich nur vermuten, dass sich Remotearbeit generell positiv auf die Produktivität und vor allem auf die Effizienz auswirkt.
Sicher zu sagen ist es jedoch nicht, da die Auswirkungen von Remotearbeit vor allem von den individuellen Personen und den jeweiligen Arbeitsaufgaben abhängt \cite{glenn_dutcher_effects_2012}, \cite{holand_homeoffice_2023}.

\section{Ökologische Aspekte}

% Auch die ökologischen Auswirkungen von Remotearbeit sind komplex und schwer abzuwägen.
% Im Folgenden werden die zwei wichtigsten Aspekte, die Ver\-änderungen im Energieverbrauch bei der Arbeit und im Pendelverkehr, behandelt.

\subsection{Energieverbrauch von Remotearbeit}

Eine Untersuchung von 2020 zeigt, dass der Energieverbrauch aufgrund des individuellen Heizens, Klimatisierens und Beleuchtens im Homeoffice zwischen 7 und 23 \% höher ist, als im Büro \cite{daniel_crow_working_2022}.
Aber auch Video-Konferenzen brauchen viel Elektrizität. 
Eine Stunde Video-Konferenz resultiert in einem CO2-Ausstoß von rund 55 bis 160 g (vergleichbar mit 250-Meter-PKW-Fahrt) \cite{deutschlandfunk_nova_internet-konferenz_2021}, \cite{eisemann_treibhauseffekt_2021}.

\subsection{Veränderungen im Pendelverkehr}

Die Veränderungen im Pendelverkehr durch Remotearbeit sind auf lange Sicht schwer abzuschätzen.
Zwar ergaben Studien zwar, dass sich der CO2-Fußabdruck ab einer Pendelstrecke von 6 km durch Remotearbeit verbessert \cite{daniel_crow_working_2022}, aber langfristig könne es zu Reboundeffekten kommen.
So könnten seltenere Fahrten ins Büro dazu führen, dass die Menschen längere Strecken in Kauf nehmen und der Anreiz für sparsame Fahrzeuge sinkt \cite{waldemar_marz_reduziert_2022}.\\\\
Wie die ökologischen Auswirkungen also langfristig aussehen, ist schwer abzuschätzen, hängt von vielen Faktoren ab und bedarf weiterer Forschung.
Annehmbar ist, dass durch die gesparten Fahrten ins Büro und die dadurch reduzierten CO2-Emissionen, Remotearbeit langfristig einen positiven Effekt auf die Umwelt hat.

\section{Soziale Auswirkungen}

\subsection{Auswirkungen auf die Work-Life-Balance der Mitarbeiter}

Ein bedeutender Aspekt der sozialen Auswirkungen von Remotearbeit ist die Veränderung der Work-Life-Balance. So gaben bei einer Befragung des WSI im Jahre 2014 52\% der Befragten an, dass sich die Vereinbarkeit von Arbeit und Privatleben durch Homeoffice verbessert hat. Davon sind jedoch 18\% der Meinung, dass die Grenzen zwischen Arbeit und Privatleben verschwimmen. \cite{wsi_wsi_2014}
\\\\
Im Jahre 2020 wurde eine Ähnliche Befragung des WSI durchgeführt. Dabei wuchs der Anteil der Befragten, die eine Verbesserung der Work-Life-Balance durch Homeoffice sehen, auf 77\%. Der Anteil der Befragten, die der Meinung sind, dass die Grenzen zwischen Arbeit und Privatleben verschwimmen, stieg jedoch im Vergleich zu 2014 sehr stark auf 60\% an. \cite{wsi_homeoffice_2020}
%TODO: Quelle für die Befragungen

% \subsection{Veränderungen in der Teamdynamik und Unternehmenskultur}

\subsection{Herausforderungen und Lösungsansätze für soziale Aspekte}

In der heutigen Arbeitswelt stehen wir vor bedeutenden sozialen Herausforderungen, insbesondere hinsichtlich der drohenden sozialen Isolation und der oft unklaren Strukturierung von Arbeitszeiten und Pausen. Studien zeigen, dass rund 50,2\% der Arbeitnehmer den fehlenden Kontakt zu Kollegen als besonders belastend empfinden. \cite{statista_belastende_2021} Um diesem Problem entgegenzuwirken, ist es entscheidend, regelmäßige Meetings einzuplanen und den Arbeitsalltag klar zu strukturieren. Durch regelmäßige Interaktionen in Meetings können nicht nur fachliche Themen besprochen, sondern auch soziale Bindungen gestärkt werden. Zudem kann eine klare Strukturierung des Arbeitstages dazu beitragen, eine ausgewogene Balance zwischen Arbeit und Erholung zu schaffen, was wiederum die soziale Gesundheit der Mitarbeiter fördert. \cite{schneider_homeoffice_2022}

\section{Fazit und Ausblick}

Beispiel für Citation\cite{noauthor_internet-konferenz_2021}
(in der Datei literature.bib)


\bibliographystyle{splncs04}
\bibliography{literature}
\end{document}

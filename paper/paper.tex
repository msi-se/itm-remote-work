\documentclass[runningheads]{llncs}
%
\usepackage{graphicx}
\usepackage[ngerman]{babel}
\usepackage{cite}
\usepackage{array}
% \usepackage{hyperref}
% \renewcommand\UrlFont{\color{blue}\rmfamily} % für hyperref

\begin{document}

\title{Digitale Transformation der Arbeitswelt: Auswirkungen von Remotearbeit auf Ökonomie, Ökologie und Soziales}

\titlerunning{Digitale Transformation der Arbeitswelt}

\author{Fabian Klimpel\inst{1}\orcidID{} \and
Tobias Tögel\inst{1}\orcidID{} \and
Johannes Brandenburger\inst{1}\orcidID{}}
%
\authorrunning{Klimpel, Tögel, Brandenburger}

\institute{HTWG - Hochschule Konstanz Technik, Wirtschaft und Gestaltung}

\maketitle

\begin{abstract}
The abstract should briefly summarize the contents of the paper in
15--250 words.

\keywords{First keyword  \and Second keyword \and Another keyword.}

\end{abstract}

\section{Remotearbeit im Überblick}

\subsection{Definition Remotearbeit}

Im Zusammenhang mit Remotearbeit unterscheidet man zwischen folgenden Begriffen:\\ 

"Home-Office": Gelegentliches Arbeiten von Zuhause ohne feste Vereinbarung zwischen Arbeitgeber und Arbeitnehmer\cite{Lenz}. Freiwilliges Angebot des Arbeitgebers\cite{Lenz}. Das BMAS arbeitet akutell an einem Gesetzesentwurf für ein Recht auf Home-Office\cite{Lenz}.\\ 

"Telearbeit (Remotearbeit)": Beschäftigte arbeiten (zum Teil) an vom Arbeitgeber fest eingerichteten Bildschirmarbeitsplatzes außerhalb des Betriebes. Seit 2016 gesetzlich geregelt in § 2 Abs. 7 ArbStättV\cite{bundestag}. Erfordert Vereinbarung über wöchentliche Arbeitszeit und Dauer der Einrichtung des Arbeitsplatzes (Individuelle oder Betriebliche Vereinbarung)\cite{vbg23}. Vereinbarungen Regeln hierbei auch die Erstattung von entstandenen Kosten der Telearbeit\cite{vbg23}. Der Arbeitgeber ist verpflichtet alle nötigen Arbeitsmittel wie beispielsweise PC, Kommunikationstechnik und Mobiliar bereitzustellen\cite{vbg23}.\\    

"Mobile Arbeit": Virtuelle Arbeit nicht gebunden an Arbeitsplatz im Büro oder Zuhause\cite{bundestag}. Arbeit an belibigen Orten (z.B im Zug, beim Kunden, etc.). Aktuell nicht ausdrücklich gesetzlich geregelt\cite{bundestag}.

\subsection{Verbreitung von Remotearbeit}
Laut Ifo Institut arbeiteten im Jahr 2023 Gesamtwirtschaftlich betrachtet 24,7\% aller Beschäftigten in Deutschland teileweise im Homeoffice\cite{ifo23}. Einen besonders hohen Anteil hat der Dienstleistungssektor mit 35,6\%\cite{ifo23}.Dienstleistungen der Informationstechnologie haben mit 73.4\% wiederrum den höchsten Anteil im Dienstleistungssektor\cite{ifo23}.\\
Entwicklungen der letzten 5 Jahre zeigen, dass durch die Covid-19 Pandemie das Arbeiten von Zuhause stark an Bedeutung gewonnen hat. Zwischen den Jahren 2019 \& 2020 hat sich der Wert von 13\% auf 21\% fast verdoppelt\cite{statis23}. Laut Ifo Institut gab es nach Aufhebung der Homeofficepflicht im März 2022 keine Veränderungen des Anteils an Beschäftigten, die teileweise im Homeoffice arbeiten\cite{ifo23}.

\section{Intergenerationale Aspekte}

\subsection{Historische Entwicklung von Remotearbeit}
Erste Telearbeitkonzepte enstanden in den 1970er Jahren in den USA\cite{Dangelmaier.1999}. Anfang 1980er taucht der Begriff Telearbeit erstmals in Deutschland auf mit Pilotprojekten bei Siemens\cite{Dangelmaier.1999}.Aufgrund der hohen ökonomische \& technologische Barrieren gewann Telearbeit erst im laufe der Zeit an Relevanz\cite{Dangelmaier.1999}.\\
In den 1990er Jahren entstanden neue Konzepte für Organisationsformen wie beispielsweise das Konzept "Virtuelle Organisation"\cite{Siedenbiedel.2020}. Durch das Entstehen und die rasante Entwicklung moderner IuK und der damit verbundenen Möglichkeiten durch Telearbeit, wurde die virtuelle Organisation erst möglich\cite{Siedenbiedel.2020}.

\subsection{Intergenerationaler Vergleich}
Bei einem Vergleich zwischen heutigen Berufseinsteigern und früheren Generationen lassen sich drastische Unterschiede in der Arbeitswelt erkennen:
\vspace{-\baselineskip}
\begin{center}
\begin{tabular}{ | m{13em} | m{13em}| m{13em} | } 
\hline
Generation (*1928-*1945) & Generation (*1946-*1964) & Generation ($>$*1996) \\
\hline
Pre-Remotearbeit & erste Generation mit Remotearbeitsmöglichkeiten & Remotearbeit Teil des Arbeits\-alltags\\
\hline
- & Geringe Rechenleistung in stationären Geräten & Große Rechenleistung in tragbaren Geräten\\
\hline
- & Verfügbarkeit des Internet ab 1993\cite{CERN} & Verfügbarkeit von Highspeed-Internet\\
\hline
- & Anfangs weltweit wenige tausend Telearbeiter\cite{heilmann1983softwareentwicklung} & in Deutschland ca. 11,5 mio. Beschäftigte, teilweise im Homeoffice (2023)\cite{statis24, ifo23}\\
\hline
\end{tabular}
\end{center}

\section{Technologische Voraussetzungen und Herausforderungen}

\subsection{Infrastruktur und Technologien für erfolgreiche Remotearbeit}

\subsection{Datenschutz und Sicherheitsaspekte}

\subsection{Umgang mit technologischen Herausforderungen}


\section{Ökonomische Auswirkungen}

\subsection{Einsparungen durch Remotearbeit}

\subsection{Produktivitätssteigerung durch Homeoffice}

\subsection{Veränderungen in der Unternehmensstruktur und -kultur}

\subsection{Wirtschaftliche Chancen und Herausforderungen}


\section{Ökologische Aspekte}

\subsection{Energieverbrauch von Online-Konferenzen}

\subsection{Veränderungen im Pendelverkehr}


\section{Soziale Auswirkungen}

\subsection{Auswirkungen auf die Work-Life-Balance der Mitarbeiter}

\subsection{Veränderungen in der Teamdynamik und Unternehmenskultur}

\subsection{Herausforderungen und Lösungsansätze für soziale Aspekte}


\section{Fazit und Ausblick}

Beispiel für Citation\cite{noauthor_internet-konferenz_2021}
(in der Datei literature.bib)


\bibliographystyle{splncs04}
\bibliography{literature}
\end{document}

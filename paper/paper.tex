% This is samplepaper.tex, a sample chapter demonstrating the
% LLNCS macro package for Springer Computer Science proceedings;
% Version 2.20 of 2017/10/04
%
\documentclass[runningheads]{llncs}
%
\usepackage{graphicx}
\usepackage[ngerman]{babel}
\usepackage{cite}
% \usepackage{hyperref}
% \renewcommand\UrlFont{\color{blue}\rmfamily}

\begin{document}

\title{Digitale Transformation der Arbeitswelt: Auswirkungen von Remotearbeit auf Ökonomie, Ökologie und Soziales}

\titlerunning{Digitale Transformation der Arbeitswelt}

\author{Fabian Klimpel\inst{1}\orcidID{} \and
Tobias Tögel\inst{1}\orcidID{} \and
Johannes Brandenburger\inst{1}\orcidID{}}
%
\authorrunning{Klimpel, Tögel, Brandenburger}

\institute{HTWG - Hochschule Konstanz Technik, Wirtschaft und Gestaltung}
%
\maketitle              % typeset the header of the contribution
%
\begin{abstract}
The abstract should briefly summarize the contents of the paper in
15--250 words.

\keywords{First keyword  \and Second keyword \and Another keyword.}
\end{abstract}


\newpage
\section{Remotearbeit im Überblick}

\subsection{Definition Remotearbeit}

\subsection{Varianten von Remotearbeit}

\subsection{Verbreitung von Remotearbeit}

\newpage
\section{Intergenerationale Aspekte}

\subsection{Historische Entwicklung von Remotearbeit}

\subsection{Vergleich zwischen Generationen}

\newpage
\section{Technologische Voraussetzungen und Herausforderungen}

\subsection{Infrastruktur und Technologien für erfolgreiche Remotearbeit}

\subsection{Datenschutz und Sicherheitsaspekte}

\subsection{Umgang mit technologischen Herausforderungen}

\newpage
\section{Ökonomische Auswirkungen}

\subsection{Einsparungen durch Remotearbeit}

\subsection{Produktivitätssteigerung durch Homeoffice}

\subsection{Veränderungen in der Unternehmensstruktur und -kultur}

\subsection{Wirtschaftliche Chancen und Herausforderungen}


\newpage
\section{Ökologische Aspekte}

\subsection{Energieverbrauch von Online-Konferenzen}

\subsection{Veränderungen im Pendelverkehr}

\newpage
\section{Soziale Auswirkungen}

\subsection{Auswirkungen auf die Work-Life-Balance der Mitarbeiter}

\subsection{Veränderungen in der Teamdynamik und Unternehmenskultur}

\subsection{Herausforderungen und Lösungsansätze für soziale Aspekte}

\newpage
\section{Fazit und Ausblick}

Beispiel für Citation\cite{noauthor_internet-konferenz_2021}
(in der Datei literature.bib)


\newpage
\bibliographystyle{splncs04}
\bibliography{literature}
\end{document}
